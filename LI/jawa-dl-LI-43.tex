
\chapter{SANDHANGAN SANDHINGAN}

\begin{center}
{\jawa ?snD=zn\pangkon snDi=zn\pangkon.}

\end{center}



\begin{center}
\begin{tabular}{ll}
{\jawa [\symbol{218}} & {\jawa [\symbol{218}o}\\
 taling& taling tarung\\
ngemu swara \'{e} & ngemu swara o \\
\end{tabular}
\end{center}



\begin{center}
Panganggone taling

\begin{tabular}{ll}
tanpa taling & mawa taling\\ \\
"papa" & "p\'{e}p\'{e}"\\
{\jawa pp} & {\jawa [p[p}\\ \\
"papat" & "p\`{e}p\`{e}t"\\
{\jawa ppt\pangkon} & {\jawa [p[pt\pangkon}\\
\end{tabular}
\end{center}

\begin{center}
Panganggone taling tarung

\begin{tabular}{ll}
tanpa taling tarung& mawa taling tarung\\ \\
"sata" & "soto'{e}"\\
{\jawa st} & {\jawa [so[to}\\ \\
"babat" & "bobot"\\
{\jawa bbt\pangkon} & {\jawa [bo[bot\pangkon}\\
\end{tabular}
\end{center}



\flushright{
 dening Agus Wepe\\ Muride Mas Djaka. Asli Wanasaba.\\ Seneng utheg sinau aksara Jawa.
}

