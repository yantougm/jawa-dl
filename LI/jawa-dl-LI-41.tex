
\chapter{SANDHANGAN TUMRAP AKSARA JAWA}

\begin{center}

{\jawa ?snD=znT|m]pHkSrjw.}

\end{center}



\begin{center}
\begin{tabular}{lll}
{\jawa siti} & $=$ & siti/lemah\\
{\jawa wizi} & $=$ & wingi\\
{\jawa pipi} & $=$ & pipi\\
{\jawa sif} & $=$ & sida\\
{\jawa lil} & $=$ & lila\\
{\jawa spi} & $=$ & sapi\\
{\jawa adi} & $=$ & adhi\\
\end{tabular}
\end{center}



\begin{center}
\begin{tabular}{lll}
{\jawa st} & $\Rightarrow$ & {\jawa siti}\\
sata & &siti\\
{\jawa wz} & $\Rightarrow$ & {\jawa wizi}\\
wanga & &wingi\\
{\jawa sf} & $\Rightarrow$ & {\jawa sif}\\
sada & &sida\\
{\jawa sp} & $\Rightarrow$ & {\jawa spi}\\
sapa & &sapi\\
\end{tabular}
\end{center}


\flushright{
 dening Agus Wepe\\ Muride Mas Djaka. Asli Wanasaba.\\ Seneng utheg sinau aksara Jawa.
}

