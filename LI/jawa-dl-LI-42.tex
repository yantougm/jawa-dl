
\chapter{SANDHANGAN TUMRAP AKSARA JAWA (bageyan 2)}

\begin{center}

{\jawa ?snD=znT|m]pHkSrjw \symbol{192}b[gyn\pangkon 3\symbol{193}.}

\end{center}



\begin{center}
\begin{tabular}{lll}
{\jawa kufu} & $\Rightarrow$ & kudu\\
{\jawa aku} & $\Rightarrow$ & aku\\
{\jawa pupu} & $\Rightarrow$ & pupu\\
{\jawa tuw} & $\Rightarrow$ & tuwa\\
\end{tabular}
\end{center}


\begin{center}
\begin{tabular}{lll}
{\jawa gel} & $\Rightarrow$ & g\^{e}la\\
{\jawa seg} & $\Rightarrow$ & s\^{e}ga\\
{\jawa asem\pangkon} & $\Rightarrow$ & as\^{e}m\\
{\jawa byem\pangkon} & $\Rightarrow$ & by\^{e}m\\
{\jawa seseg\pangkon} & $\Rightarrow$ & s\^{e}s\^{e}g\\
\end{tabular}
\end{center}


\flushright{
 dening Agus Wepe\\ Muride Mas Djaka. Asli Wanasaba.\\ Seneng utheg sinau aksara Jawa.
}

