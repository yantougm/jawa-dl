\documentclass{article}
\usepackage{longtable}
\usepackage{fontspec}
\usepackage{amssymb}
\usepackage{lmodern}
\usepackage{palatino}
\usepackage{microtype}
\input{jawamac}
\usepackage{luacode}
\parskip6pt
%\parindent0pt
\newcommand\tstrut{\rule{0pt}{2.6ex}} % top strut of height 2.6ex to add space after \hline

\newcommand{\jawa}{\fontspec{Genk Kobra Gerbangpraja}[]}
\newcommand{\jawauc}{\large\fontspec{Nawatura beta}[]}
\setmainfont{CormorantGaramond}

\begin{document}
\begin{center}

\Large{\textbf{DIPEPET APA ORA}}

{\jawa ?fipepetHp[aor.}

\end{center}



\begin{center}
\begin{tabular}{cc}
{\jawa sd=} & {\jawa sed\_} \\
sadhang& sedheng 
\end{tabular}
\end{center}



\begin{center}
\begin{tabular}{cc}
{\jawa x[bo} & {\jawa finTenRe[bo} \\
Rebo& dinten Rebo\\ \\
{\jawa Xgi} & {\jawa se[nnLegi} \\
Legi& Senen Legi
\end{tabular}
\end{center}

\begin{center}
cakra keret

wanda re minangka seselan


{\jawa gum\}beg\pangkon} 
 
"gumrebeg"

pasangan ra + pepet

wanda re miwiti tembung anyar


{\jawa afusResik\pangkon} 
 
"adus resik"

\end{center}

\flushright{
 dening Agus Wepe\\ Muride Mas Djaka. Asli Wanasaba.\\ Seneng utheg sinau aksara Jawa.
}
\end{document}

