\chapter{PEPET, WULU, LAN CECAK}


\begin{center}
{\jawa ?pepet\pangkon wulu lnCeck\pangkon.}

\end{center}



\begin{center}
\begin{tabular}{cc}
{\jawa \symbol{218}e} & {\jawa \symbol{218}i} \\
pepet & wulu\\
bunderane menga & bunderane nutup\\
\end{tabular}
\end{center}


\begin{center}
\begin{tabular}{cc}
{\jawa \symbol{218}\_} & {\jawa \symbol{218}i=} \\
pepet + cecak & wulu + cecak\\
cecak ana jero & cecak ana jaba\\
\end{tabular}
\end{center}


\begin{center}
\begin{tabular}{cc}
{\jawa \symbol{218}i} & {\jawa \symbol{218}\symbol{"0BB}} \\
wulu mligi & wulu isi cecak\\
swara \textit{i} cendhak & swara \textit{i} dawa\\
\end{tabular}
\end{center}



\flushright{
 dening Agus Wepe\\ Muride Mas Djaka. Asli Wanasaba.\\ Seneng utheg sinau aksara Jawa.
}

