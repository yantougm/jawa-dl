\documentclass{article}
\usepackage{longtable}
\usepackage{fontspec}
\usepackage{amssymb}
\usepackage{lmodern}
\usepackage{palatino}
\usepackage{microtype}
../jawamac.tex
\usepackage{luacode}
\parskip6pt
%\parindent0pt
\newcommand\tstrut{\rule{0pt}{2.6ex}} % top strut of height 2.6ex to add space after \hline

\newcommand{\jawa}{\fontspec{Genk Kobra Gerbangpraja}[]}
\newcommand{\jawauc}{\large\fontspec{Nawatura beta}[]}
\setmainfont{CormorantGaramond}

\begin{document}
\begin{center}

\Large{\textbf{PEPET, WULU, LAN CECAK}}

{\jawa ?pepet\pangkon wulu lnCeck\pangkon.}

\end{center}



\begin{center}
\begin{tabular}{cc}
{\jawa \symbol{218}e} & {\jawa \symbol{218}i} \\
pepet & wulu\\
bunderane menga & bunderane nutup\\
\end{tabular}
\end{center}


\begin{center}
\begin{tabular}{cc}
{\jawa \symbol{218}\_} & {\jawa \symbol{218}i=} \\
pepet + cecak & wulu + cecak\\
cecak ana jero & cecak ana jaba\\
\end{tabular}
\end{center}


\begin{center}
\begin{tabular}{cc}
{\jawa \symbol{218}i} & {\jawa \symbol{218}\symbol{"0BB}} \\
wulu mligi & wulu isi cecak\\
swara \textit{i} cendhak & swara \textit{i} dawa\\
\end{tabular}
\end{center}



\flushright{
 dening Agus Wepe\\ Muride Mas Djaka. Asli Wanasaba.\\ Seneng utheg sinau aksara Jawa.
}
\end{document}

