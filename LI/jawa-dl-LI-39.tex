\documentclass{article}
\usepackage{longtable}
\usepackage{fontspec}
\usepackage{amssymb}
\usepackage{lmodern}
\usepackage{palatino}
\usepackage{microtype}
\input{jawamac}
\usepackage{luacode}
\parskip6pt
%\parindent0pt
\newcommand\tstrut{\rule{0pt}{2.6ex}} % top strut of height 2.6ex to add space after \hline

\newcommand{\jawa}{\fontspec{Genk Kobra Gerbangpraja}[]}
\newcommand{\jawauc}{\large\fontspec{Nawatura beta}[]}
\setmainfont{CormorantGaramond}

\begin{document}
\begin{center}

\Large{\textbf{PANGKON MANING, PANGKON MANING}}

{\jawa ?p=[konMni=,p=[konMni=.}

\end{center}



\begin{center}
\textbf{Tuladha\'{e} pada lingsa lan pada lungsi}

{\jawa ?[ao[l}{\jawauc \symbol{43395}}{\jawa [aa=g/wwisPi[nDo,}

{\jawa [f[nput][nmu=siji.} 

\textbf{Ol\`{e}h\'{e} nggarwa wis pindho, d\'{e}n\'{e} putran\'{e} mung siji.}

\textit{[adeg--adeg] ol\`{e}h\'{e} nggarwa wis pindho [lingsa] d\'{e}n\'{e} putran\'{e} mung siji [lungsi]}
\end{center}

\begin{center}
\textbf{Pangkon minangka lirun\'{e} lingsa lan lungsi}

{\jawa ?put][nnenem\pangkon ln=[lo[ro,w[fonPpt\pangkon,} 

{\jawa mulu/}
\end{center}




\flushright{
 dening Agus Wepe\\ Muride Mas Djaka. Asli Wanasaba.\\ Seneng utheg sinau aksara Jawa.
}
\end{document}

