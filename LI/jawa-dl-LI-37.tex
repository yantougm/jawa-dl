\chapter{YEN WIS DUWE PASANGAN}
\begin{center}


{\jawa ?[yniWsF|[wps=zn\pangkon.}
\end{center}



\begin{center}
{\jawa \symbol{164}\pangkon $+$ a $=$  \symbol{164}H}

{\large tuladhan\'{e}}

\begin{tabular}{cccccc}
{\jawa wels\pangkon} & $+$ & {\jawa asih }&$=$&\multicolumn{2}{c}{\jawa welsHsih}\\
welas & $+$ & asih &$=$ &\multicolumn{1}{c}{welas}&  \multicolumn{1}{c}{asih} \\
\end{tabular}
\end{center}



\begin{center}

{\large pasangan $+$ sandhangan}

\begin{tabular}{cc}
{\jawa [amP-k\pangkon} & {\jawa cnF`}\\
\'{e}mpyak &  candra\\
\end{tabular}
\end{center}

\begin{center}

{\large pasangan $+$ pasangan}

\begin{tabular}{cc}
{\jawa kve}{\jawauc \symbol{57367}\symbol{57402}}{\jawa b\pangkon} & {\jawa sukSM}\\
kancleb &  suksma\\
\end{tabular}
\end{center}


\flushright{
 dening Agus Wepe\\ Muride Mas Djaka. Asli Wanasaba.\\ Seneng utheg sinau aksara Jawa.
}

