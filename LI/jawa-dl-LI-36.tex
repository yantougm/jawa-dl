\documentclass{article}
\usepackage{longtable}
\usepackage{fontspec}
\usepackage{amssymb}
\usepackage{lmodern}
\usepackage{palatino}
\usepackage{microtype}
\input{jawamac}
\usepackage{luacode}
\parskip6pt
%\parindent0pt
\newcommand\tstrut{\rule{0pt}{2.6ex}} % top strut of height 2.6ex to add space after \hline

\newcommand{\jawa}{\fontspec{Genk Kobra Gerbangpraja}[]}
\setmainfont{CormorantGaramond}

\begin{document}
\begin{center}

\Large{\textbf{DIPANGKU BEN MENENG}}

{\jawa ?fip=ku[bnMen\_}
\end{center}

"Dhi, dhek cilik samang nyandhing Bapak Ibu 'pa 'ra?" Mangkono pitakone Mas Djaka minangka pambukaning obrolan. 

"Lha iya, ta, Mas," aku mangsuli kanthi mantep. 

"Ora, maksudku, rak ora diemong Simbah, apa sapa, ngono," ujare Mas Djaka sinambi orek-orek ana kertas. "Ateges ya ngrasakake digendhong, dipangku bapak-ibune.

" Ho'oh, Mas, kala-kala ya karo Simbah," jawabku. "Dhek biyen aku cilik, gonta-ganti di gendhongi kambi Simak, Ramak, apa Simbah, dikekudang mawa panggadhang kang apik--apik. Wektu semana aku durung duwe panemu apa--apa, awit ngertiku ya mung kekarepanku tansah disembadani. Sawayah--wayah wong tuwaku tindak ngendi ya aku tansah ana gendhongan. Mula dadi bocah cilik ya anane mung manut, dadi durung bisa nggugu karepe dhewe. Malah-malah .." 

"Nyoh Dhi," Mas Djaka nuduhake orek-orekane mau, "Iki sing diarani pangkon." Aku ngematake sambi manthuk-manthuk. Ana simbul kang wangune kaya kursi mawa lendhehan. 

Mas Djaka nuli nerangake yen pangkon bisa diupamakake pamangkoning wong atuwa tumrap bocahe. Karepe wong tuwa, yen bocah nangis, janji wis dipangku, digendhong apa dikekudang, bisa dadi sarana lerem lan menenge bocah mau. Semana uga bab tandha pangkon, yen wis ditepakake ana sawijining aksara, bisa kanggo ngenengake swarane aksara kang kasinungan. Tuladhane yen aksara \textit{ka} dipangku, swarane \textit{a} banjur ilang, dadi mung mligi konsonan tanpa vokal.

\begin{center}
\begin{tabular}{ll}
{\jawa k} &ka\\
{\jawa k\pangkon} &k\\
{\jawa bpk} &bapaka \\
{\jawa bpk\pangkon} &bapak \\
\end{tabular}
\end{center}

Saiki 'mang cobanen, nulis nganggo pangkon." 

Barang dakjajal, Mas Djaka malah mesam--mesem, sajak ana sing luput saka tulisanku. Pirsa yen aku bingung, Mas Djaka banjur nerangake. "Ora, dudu apa--apa kok, Dhi. Iki ora luput blas. Wis bener, nanging kalebune boros." Lah pripun, ta? 

\textit{Jebulnya-jebul}, olehku nulis samubarang tembung kang ngemu swara mati, sakabehane dakpangkoni. Mangka panganggoning pangkon iku kalebu mirunggan, yaiku yen kepepet wis kepanjingan pasangan matumpuk-tumpuk, kanggo mrenahake jeneng wong utawa panggonan, lan sapanunggalane. Liyane kuwi, kang dienggo cukup pasangan wae.

\begin{center}
\begin{tabular}{ll}
\multicolumn{2}{c}{\'{e}ntuk telahan "Bagong"}\\
{\jawa [an\pangkon tuk\pangkon telan\pangkon b[g=o  }  & (kl\`{e}ru)\\
{\jawa [an\psgta{}|kTelhan\pangkon b[g=o } & (baku)\\
\end{tabular}
\end{center}

"Kaya aku iki, Dhi, kang wis duweni pasangan," swarane Mas Djaka malih dadi lirih, " tarkadhang ya isih bisa kepepet, banjur sambat ana pangkone wong tuwa." 

Lamat--lamat aku krungu swara wong dhehem--dhehem saka pawon. 

"Nanging sing kaya mangkonno aja nganti dibacut-bacutake ya, Dhi. Mundhak boros tur aleman." Swarane Mas Djaka sangsaya lirih. Lah kok malah aku sing menjep.

\flushright{
 dening Agus Wepe\\ Muride Mas Djaka. Asli Wanasaba.\\ Seneng utheg sinau aksara Jawa.
}
\end{document}

