\documentclass{article}
\usepackage{longtable}
\usepackage{fontspec}
\usepackage{amssymb}
\usepackage{lmodern}
\usepackage{palatino}
\usepackage{microtype}
../jawamac.tex
\usepackage{luacode}
\parskip6pt
%\parindent0pt
\newcommand\tstrut{\rule{0pt}{2.6ex}} % top strut of height 2.6ex to add space after \hline

\newcommand{\jawa}{\fontspec{Genk Kobra Gerbangpraja}[]}

\begin{document}
\begin{center}

\Large{\textbf{NYANDHANG SAPRELUNE}}

\fontspec{Genk Kobra Gerbangpraja}[]
?vnQ=sp\}lu[n. 

\end{center}

Kaya kang wis dakcritakake minggu kapungkur, dina iki aku sida nyelakake sowan Mas Djaka saprelu ajar nulis Jawa. Olehe milih dina Setu amarga tiba preine aku lan Mas Djaka anggone nyambutgawe. Dilalah amarga \textit{weekend}, wektune ngepasi tanggane Mas Djaka duwe gawe supitan. Mangka swarane speaker gumrebeg kaya gunung njeblug. Mula Mas Djaka banjur ngejak sinau ana kafe anyar tengah sawah kae, ngiras penglaris karang sing tuku ya durung rame banget. 

Ing wulangan kang sepisan iki, Mas Djaka nedya mulangake bab sandhangan kang minangka panyalining swara ing saben wanda. Jinising sandhangan swara yaiku ana \textit{pepet, wulu, suku, taling}, lan \textit{taling-tarung}. Ngendikane Mas Djaka, panganggone sandhangan swara kudu jumbuh kalawan jejeg--miringing swara, supaya ora gawe bingung tumrap sapa wae kang maca. Ewadene isih akeh wong kang meksa kleru anggone ngetrapake sandhangan, contone ing bab panulising tembung "kucing". Kang kudune swara  \textit{miring} nganggo sandhangan wulu, nanging malah disandhangi taling amarga ungele memper swara \'{e}.

\begin{center}
\begin{tabular}{lll}
{\jawa ku[c=} & kuc\'{e}ng \texttimes & (kl\`{e}ru)\\
{\jawa kuci=} & kucing \checkmark& (baku)\\
{\jawa mu[l/o} & mulor \texttimes & (kl\`{e}ru)\\
{\jawa mulu/} & mulur \checkmark& (baku)\\
\end{tabular}
\end{center}

"Galo sawangen,Dhi." Mas Djaka nglirik marang ibu--ibu kang lagi matun ing sawah kae. Ingatase lagi nyambut gawe reged, nanging isih meksa nganggo mas--masan gumebyar pating cranthel saawak. Senajan mas--masan iku duweke dhewe, nanging yen disawang tetep wae ora jumbuh karo kahanan, awit sing jenenge kuningan kuwi ora bisa disandhang sawayah-wayah, apa maneh yen pinuju ing wayahe nyawah. Malah--malah, yen wayahe adus ya ora prelu nyandhang babar pisan. Panganggone sandhangan ing aksara Jawa uga bisa diupamakakekaya mangkono. Mula aksara kang ora nganggo sandhangan kuwi bisa diarani aksara nglegena, ateges ligan. 

Mas Djaka nuli crita yen ana wong kang nulis jeneng kutha "Wonosobo" nganggo taling--tarung, adhedhasar ejaan Latine kang pancen nganggo huruf o. Mangkono kuwi uga kleru, jalaran swara o ing kono yen kaprenahake sajatine swara o kang diwaca jejeg, mula cukup ditulis nglegena tanpa taling-tarung. Nyatane, tumrap para sedu- lur Panginyongan (Banyumasan), tembung "Wo- nosobo" pocapane isih nganggo a \textit{miring}, mula keprungune dadi kaya wanas-sabak. Tuladha liyane, upamane yen tembung kuwi kawuwuhan panambang-\textit{e}/-\textit{ne}, swarane a jejeq mesthi bakal kasalin a \textit{miring}. Mula, tulisan "Wanasabane ora kena diwaca \textit{wonosobone}, ananging wonosabane, awit dhasare ya pancen ngemu swara a, amung beda jejeg--miringe wae.

\begin{center}
\begin{tabular}{lll}
{\jawa [wo[no[so[bo} & wonosobo \texttimes & (kl\`{e}ru)\\
{\jawa wnsb} & wanasaba \checkmark& (baku)\\
\end{tabular}
\end{center}

"Mula kadhang kala, nglegena kuwi ora salawase saru, apa maneh yen sandhangane malah meksa gawe bingung," Mas Djaka ngobahake racikan saemper tandha kutip. Aku kukur sirah senajan ora gatel.

\flushright{
 dening Agus Wepe\\ Muride Mas Djaka. Asli Wanasaba.\\ Seneng utheg sinau aksara Jawa.
}
\end{document}

