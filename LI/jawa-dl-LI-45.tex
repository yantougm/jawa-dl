
\chapter{GEDHE CILIK ANA GUNANE}

\begin{center}
{\jawa ?g[dcilikHngun[n.}

\end{center}



\begin{center}
Aksara murda

\begin{tabular}{ccccccc}
{\jawa !} & {\jawa @}& {\jawa \#}& {\jawa \$}& {\jawa \%}& {\jawa \&}& {\jawa *}\\
na & ka & ta & sa & pa & ga & ba\\ \\
\multicolumn{7}{l}{Pasangane}\\
{\jawa \symbol{218}\symbol{174}} 
& {\jawa \symbol{218}\symbol{175}} 
& {\jawa \symbol{218}\symbol{176}} 
& {\jawa \symbol{218}\symbol{177}} 
& {\jawa \symbol{218}\symbol{178}} 
& {\jawa \symbol{218}\symbol{180}} 
& {\jawa \symbol{218}\symbol{181}} 
\\
na & ka & ta & sa & pa & ga & ba\\ 
\end{tabular}
\end{center}



\begin{center}
\begin{tabular}{lll}
{\jawa ms\pangkon j@} & Mas Jaka\\
{\jawa \$u/yni=z]t\pangkon} &  Suryaningrat\\
{\jawa \$urby} & Surabaya\\
{\jawa fi\%!gr} &  Dipanagara\\
\end{tabular}
\end{center}


\flushright{
 dening Agus Wepe\\ Muride Mas Djaka. Asli Wanasaba.\\ Seneng utheg sinau aksara Jawa.
}

